\section{公式和中文字体\ \ 和谐共处}

公式和中文字体和谐共处。

\subsection{LaTeX公式}

%********************正文部分********************
\begin{align*}
S=\iint\limits_{\Sigma}1 \ ds &= \int_0^{\pi}d\theta \int_0^{2\pi} r^2sin(\theta) d\phi \\
&= \int_0^{\pi}d\theta \int_0^{2\pi}\sin (\theta ) \left(\frac{1}{5} \sin (\theta  m) \sin (n \phi )+1\right)^2d\phi\\
&=\frac{4 \sin (\pi  m) \sin ^2(\pi  n)}{5 n-5 m^2 n}-\frac{\left(8 m^2+\cos (2 \pi  m)-1\right) \sin (4 \pi  n)}{200 \left(4 m^2-1\right) n}+\frac{\pi  \left(8 m^2+\cos (2 \pi  m)-1\right)}{50 \left(4 m^2-1\right)}+4 \pi\\
&=  \left(\frac{8 m^2}{50 \left(4 m^2-1\right)}+4\right)\pi
\end{align*}
%********************正文部分*******************

\subsection{中文字体设置}

默认就是宋体。

调用加粗宋体:\textbf{黑体写在这里}

调用楷体:\textit{楷体写在这里}

调用仿宋:\textsl{仿宋写在这里}

调用黑体:\heiti{黑体写在这里}

\subsection{对齐方式}


%********************排版部分********************
\begin{center} 
居中文本第一行\\
居中文本第二行\\
\end{center}

\begin{flushright}
右对齐第一行\\
右对齐第二行\\
\end{flushright}

%********************排版部分********************


\section{图片与TeX子文件\ \ 信手拈来}

图片与TeX子文件信手拈来。

\subsection{图片}

%********************图片部分********************
\begin{figure}
    \centering
    \includegraphics[scale=0.4]{include_picture/picture.jpg}
    \caption{插入图片}
\end{figure}
%********************图片部分********************

\subsection{引用Tex子文件}


%********************引用Tex子文件部分********************
\input{include_tex/include_text} %使用input不分页
%\include{includetex} %使用include将分页
%********************引用Tex子文件部分********************


%\titleformat*{\subsubsection}{\scshape\MakeLowercase}


\clearpage
\section{表格\ \ 提升逼格}

搞科研怎么能没有表格?\\

\subsection{表格}

%********************表格部分********************
关于表格的各种样式,请使用Google大法。\\
\begin{table}[H]
\caption{设置表格总长} 
\begin{tabular*}{12cm}{lll}
\hline  
Start & End  & Character Block Name \\  
\hline  
3400  & 4DB5 & CJK Unified Ideographs Extension A \\  
4E00  & 9FFF & CJK Unified Ideographs \\  
\hline  
\end{tabular*} 
\end{table} 
%********************表格部分********************


%********************代码部分********************

\section{代码片\ \ 程序员的最爱}
代码片永远是程序员的最爱,支持语法高亮,用法不妨Google。\\
\lstset{language=C}
\begin{lstlisting}
#include<iostream>
using namespace std;
int main()
{
    return 0;
}
\end{lstlisting}

%********************代码部分********************